\documentclass[../photoevap.tex]{subfiles}
\begin{document}

\section{Model}
\label{sec:model}

The viscous growth and the dynamical truncations were modeled in the same way as in FCR2018. In sections 1 and 2 we present a summary of the implementations. In section 3 we present the new implementations for photevaporation developed for the present work.

\subsection{Viscous growth of disks}

We base our description of the disks on the similarity solutions of Lynden-Bell and Pringle 1974, and the implementation of Hartmann1998. We characterize our disks by three values: their characteristic radius, their mass, and their accretion rate onto their host star.

We use values of gamma=1 and alpha=1E-4 as argued in FCR2018. This gives us realistic viscous timescales for the disks.

In this occasion we consider the characteristic radius as the radius inside which 95\% of the disk's mass resides. This, to make a closer comparison with observations (cite same reference than in my previous paper).

\subsection{Dynamical truncations}
Dynamical truncations are the same as in the previous paper. Defined by SPZ. We also use the mass from X.

\subsection{Photoevaporation}
External photoevaporation of protoplanetary disks is dominated by FUV radiation. We follow \citet{adams_photoevaporation_2004} 


We modeled only the external photoevaporation of the disk. We consider internal photoevaporation to be negligible for our analyses.

We model external photoevaporation as a mass loss that affects the stellar disks. This mass loss was calculated using the FRIED grid. The FRIED grid is a grid which calculates FUV photoevaporative mass losses for a big parameter space of disks. 

\subsubsection{Getting the FUV luminosity of the stars}
We need to calculate the FUV luminosity of each of the massive stars. For this, we use the UVBLUE spectral library. We choose all our stars to have solar metallicity. We take the ZAMS temperature of each massive star.

We calculate the ZAMS luminosity of the stars. From this we choose a spectrum, and we fit it. This allows us to get a stellar mass vs FUV luminosity relationship, which is what we use to calculate the FUV luminosity of each star all through our simulation. The fit can be seen in Figure X. The values obtained for the fit can be seen in Table Y.

External photoevaporation of the disks is a function not only of the disk properties, but also of the radiation field in which the disk is immersed. tion not only of the disk properties, but also of the radiation field in which the disk is immersed. 

\subsection{Evolution of the disks}
For each simulation we did the following:

-Start a Plummer sphere with N number of stars and virial size N

-Select all the stars under 1.9 MSun (because of the FRIED grid parameter space). To all these stars we gave circumstellar disks.

-For all the stars > 1.9 MSun (a.k.a. massive stars) we implemented stellar evolution.

-We used the UVBLUE spectral library to calculate the FUV luminosity of each of these massive stars.

-On each timestep we calculate the positions of the stars, and the radiation that each small stars gets from all the massive stars, in the distance

-This gives us the total photoevaporative mass loss rate for each small star

-We update the size of the circumstellar disk according to the viscous growth and to the photoevaporative mass loss rate