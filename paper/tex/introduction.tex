\documentclass[../photoevap.tex]{subfiles}
\begin{document}

\section{Introduction}
\label{sec:introduction}

Circumstellar disks are formed shortly after star formation. During their early years, they are immersed in the environment of star formation; that is, surrounded by left over gas and prone to dynamical interactions with other, nearby stars. 
The environment in which circumstellar disks are born can have an important impact on their evolution. Some of the mechanisms affecting their development correspond to their intrinsic viscous growth, as well as external agents such as truncation by dynamical encounters, external photevaporation, face-on accretion, and ram pressure stripping. 

It is interesting to look into the evolution of circumstellar disks inside star cluster. Our very own Sun was born within a star cluster, so understanding how the cluster environment affects the evolution of the disks can help us understand the origins of our own Solar System. Previous work has been done to study the effects of viscous growth (me, Rossoti2014) and dynamical truncations on circumstellar disks inside star clusters (spz, pfalzner). This has been done both semi-analytically and numerically. 

Work has also been done in studying both the internal and external photoevaporation of circumstellar disks. Inside star clusters, it is likely that the disks will be affected by radiation from bright stars. Work by ... and ... shows that disks are especially sensitive to FUV radiation from OB stars.

This problem is very expensive to run numerically...

In this work we present a semi analytical modelling of protoplanetary disks subject to viscous growth, dynamical truncations, and external photevaporation. We expand on the work by ME2018, where viscous growth and truncations were modelled semi analytically. We use the developments of X and Y to parametrize external photevaporation as a mass loss rate on the disk.

Our previous work showed that considering only viscous growth and dynamical truncations yields disks that are too large to be compared with current observations of protoplanetary disks. Dynamical truncations by themselves are not important enough to account for the sizes of the observed disks. There must be some other mechanisms that generates the compact disks we observe in star clusters and young star forming regions.