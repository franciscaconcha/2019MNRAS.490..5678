% mnras_template.tex 
%
% LaTeX template for creating an MNRAS paper
%
% v3.0 released 14 May 2015
% (version numbers match those of mnras.cls)
%
% Copyright (C) Royal Astronomical Society 2015
% Authors:
% Keith T. Smith (Royal Astronomical Society)

% Change log
%
% v3.0 May 2015
%    Renamed to match the new package name
%    Version number matches mnras.cls
%    A few minor tweaks to wording
% v1.0 September 2013
%    Beta testing only - never publicly released
%    First version: a simple (ish) template for creating an MNRAS paper

%%%%%%%%%%%%%%%%%%%%%%%%%%%%%%%%%%%%%%%%%%%%%%%%%%
% Basic setup. Most papers should leave these options alone.
\documentclass[fleqn,usenatbib]{mnras}

% MNRAS is set in Times font. If you don't have this installed (most LaTeX
% installations will be fine) or prefer the old Computer Modern fonts, comment
% out the following line
%\usepackage{newtxtext,newtxmath}
% Depending on your LaTeX fonts installation, you might get better results with one of these:
%\usepackage{mathptmx}
%\usepackage{txfonts}

% Use vector fonts, so it zooms properly in on-screen viewing software
% Don't change these lines unless you know what you are doing
\usepackage[T1]{fontenc}
\usepackage{ae,aecompl}
\usepackage[utf8]{inputenc}


%%%%% AUTHORS - PLACE YOUR OWN PACKAGES HERE %%%%%

% Only include extra packages if you really need them. Common packages are:
\usepackage{graphicx}	% Including figure files
\usepackage{amsmath}	% Advanced maths commands
\usepackage{amssymb}	% Extra maths symbols
\usepackage{multirow}
\usepackage{txfonts}
\usepackage[utf8]{inputenc}
\usepackage{float}
\usepackage{subcaption}
\usepackage[T1]{fontenc}
\usepackage{microtype}

\usepackage{subfiles} % to split the paper into different files
\usepackage{xcolor}
\usepackage[decimalsymbol=., expproduct=times, separate-uncertainty=true, multi-part-units=single]{siunitx} % SI units 
\DeclareSIUnit\erg{erg}

%%%%%%%%%%%%%%%%%%%%%%%%%%%%%%%%%%%%%%%%%%%%%%%%%%

%%%%% AUTHORS - PLACE YOUR OWN COMMANDS HERE %%%%%

% Please keep new commands to a minimum, and use \newcommand not \def to avoid
% overwriting existing commands. Example:
%\newcommand{\pcm}{\,cm$^{-2}$}	% per cm-squared
\newcommand\note[1]{{\colorbox{yellow!60}{\color{magenta}#1}}}

%%%%%%%%%%%%%%%%%%%%%%%%%%%%%%%%%%%%%%%%%%%%%%%%%%

%%%%%%%%%%%%%%%%%%% TITLE PAGE %%%%%%%%%%%%%%%%%%%

% Title of the paper, and the short title which is used in the headers.
% Keep the title short and informative.
\title[External photoevaporation of disks in young star clusters]{External photoevaporation shapes the distributions of circumstellar disk sizes in young star clusters}

% The list of authors, and the short list which is used in the headers.
% If you need two or more lines of authors, add an extra line using \newauthor
\author[Concha-Ramírez et al.]{
Francisca Concha-Ramírez\thanks{E-mail: fconcha@strw.leidenuniv.nl},
Martijn Wilhelm,
Simon Portegies Zwart
\\
% List of institutions
% List of institutions
%$^{1}$
Leiden Observatory, Leiden University, PO Box 9513, 2300 RA Leiden, The Netherlands\\
}

% These dates will be filled out by the publisher
\date{Accepted XXX. Received YYY; in original form ZZZ}

% Enter the current year, for the copyright statements etc.
\pubyear{2018}

% Don't change these lines
\hypersetup{draft}
\begin{document}
\label{firstpage}
\pagerange{\pageref{firstpage}--\pageref{lastpage}}
\maketitle

% Abstract of the paper
\begin{abstract}
Circumstellar disks develop as a result of the star formation process, and during their first million years of evolution they are immersed in surroundings with high gas and stellar densities. This environment is unfavorable for disk survival: they can be truncated by dynamical encounters, evaporated by the radiation of bright OB stars nearby, shrunk by ram pressure stripping, and diminished by stellar winds and supernovae explosions. Circumstellar disks are also subject to viscous spreading, a consequence of angular momentum conservation. In the present work we used numerical and semi-analytical approaches to quantify the effects of dynamical truncations, external photoevaporation, and viscous evolution of circumstellar disks inside star clusters. We ran simulations of stellar clusters with $N=$\note{X} stars in a Plummer sphere with virial radius \note{Y pc}. We evolved these clusters for \note{Z} Myr, and analyzed the mean disk lifetimes at the end of the simulations. \note{We find that...}
\end{abstract}

% Select between one and six entries from the list of approved keywords.
% Don't make up new ones.
\begin{keywords}
key 1 key 2
\end{keywords}

\section{Introduction}
\label{sec:introduction}

Circumstellar disks develop as a result of the star formation process \citep{williams2011}. Star forming regions less than $\SI{2}{Myr}$ old have been observed to have protoplanetary disks in $\sim60\%$ of their stars \citep{fedele2010}. \note{Since at least a non negligible fraction of stars are not born in isolation} \citep{bressert2010,king2012}, and gas left over from the star formation process can linger for a few Myr \citep{goodwin2009,portegieszwart2010}, the disks remain embedded in this environment during their first stages of evolution. These conditions can be hostile for the disks in a myriad of ways. The surrounding gas can shrink the disks through ram pressure stripping and face-on accretion \citep{wijnen2016, wijnen2017}. Regarding the stellar dense environment, disks can be subject to dynamical truncations \citep{vincke2015,portegieszwart2016,vincke2016} or be affected by processes related to stellar evolution, such as stellar winds \citep{pelupessy2012}, supernovae explosions \citep{close2017}, and photoevaporation due to bright OB stars in the vicinity \citep[e.g.][]{guarcello2016,haworth2017}. Since planet formation related processes seem to start very quickly in circumstellar disks ($< 0.1 - \SI{1}{Myr}$, \citet{najita2014,manara2018}), understanding the mechanisms that affect disk evolution is directly connected with understanding planetary system formation and survival. Our own Sun was born within a star cluster \citep{portegieszwart2009}, so discerning how the cluster environment affects the evolution of the disks can help us comprehend the origins of our home, the Solar System. 

There are observational indications that the environment in which the disks are surrounded just after their formation is unfavorable for their survival. Circumstellar disks have been observed to be evaporating in several star forming regions \cite[e.g.][]{fang2012,dejuanovelar2012,mann2014}. Moreover, observations suggest that disk fractions in cluster environments decline by a factor of $\sim2$ in regions close to an O-type star \citep{balog2007,guarcello2007,guarcello2009,fang2012,guarcello2016}. \citet{fatuzzo2008} estimated an FUV radiation field of up to $G_0 \approx 1000$ in star clusters of $N > 1000$ stars\footnote{$G_0$ is the FUV field in Habing units,  $\SI{1}{G_0} = \SI{1.6e-3}{\erg\per\second\per\square\cm}$. The solar vicinity is estimated to have a background FUV field of $G_0 = 1$ \citep{habing1968,parravano2003}.}, while \citet{facchini2016} showed that disks of sizes $\sim\SI{150}{au}$ are subject to photoevaporation even in very low FUV fields ($G_0 = 30$). Observational evidence of the imprints of dynamical truncations has been reported in regions of high stellar density \citep{olczak2008,reche2009,dejuanovelar2012}. Tidal stripping that can be explained by disk-star interaction has been observed in the RW Aurigae system \citep{cabrit2006,dai2015}. There is evidence that our own Solar System might have been affected by a close encounter with another star during its early stages \citep{jilkova2015,pfalzner2018}. It has also been observed that disks in dense regions are generally smaller than the ones in sparser regions \citep{clarke2007,dejuanovelar2012,mann2014}, and it is still up for debate if this is a consequence of dynamical encounters or external photoevaporation, or a combination of both. 

Circumstellar disks are not only affected by external mechanisms, but also by their own viscous growth. The combination of outwardly decreasing angular velocity, together with outwardly increasing angular momentum, causes shearing forces inside the disks. These forces induce the transport of angular momentum from the inner regions, locally lower in angular momentum, to the outer parts, which have higher local angular momentum. As a consequence mass is accreted from the innermost regions of the disk into its host star, while the outermost regions expand \citep{lynden-bell1974}. Observational reports of viscous spreading of disks around T Tauri stars have been described by \citet{isella2009} and \citet{guilloteau2011}. 

From a computational perspective, different approaches have been implemented to assemble these processes together and study their effects on the lifetimes of circumstellar disks. External photoevaporation has been modeled with radiation hydrodynamics codes that solve the flow equations through the disk boundaries, together with photodissociation region (PDR) codes to obtain the temperature profiles of the disks \citep[e.g.][]{haworth2016,facchini2016}. This approach has also been coupled with $\alpha$-disk models to account for viscous spreading \citep[e.g.][]{adams2004,anderson2013,gorti2015,rosotti2017}. The UV radiation fields to which the disks would be subject to in a clustered environment have been estimated from observations and used to calculate the PDR around the hydrodynamic disk. \citet{haworth2018a} introduced the concept of pre-computing photevaporative mass losses in terms of the surface density of the disks, an approach that we expand on in section \ref{photoevaporation}. 

Computational analyses of the impact of dynamical truncations have also been carried out in different ways. The effects of close encounters on a single N-body disk of test particles have been investigated in several studies \citep[e.g.][]{breslau2014,jilkova2016,bhandare2016,pfalzner2018}. In these cases, viscosity is usually neglected since the time scale for a dynamical encounter is much shorter than the viscous time scale for disk spreading. \citet{winter2018,winter2018a} used a ring of test particles around a star to obtain linearized expressions of the effect of a stellar perturber, and then used them to model an SPH disk including viscous spreading. Another approach, in which the dynamics of the cluster are evolved separately and the effects of dynamical encounters over the disks are calculated \textit{a posteriori}, has also been used for this problem \citep[e.g.][]{olczak2006,olczak2010,malmberg2011,steinhausen2014,vincke2015,vincke2016,vincke2018}. Directly adding SPH disks to a simulation of a massive star cluster is extremely computationally expensive. The closest effort corresponds to the recent work by \citet{rosotti2014}, in which individual SPH codes were coupled to each one of the $\SI{1}{M_\odot}$ stars in a small ($N=100$) cluster. This allowed them to study the effects of viscous spreading of the disks and dynamical truncations in a self-consistent way, but were clearly limited by the computational resources needed for this problem. Parametrized approaches have also been developed, where the cluster dynamics and effects of truncations \citep{portegieszwart2016} and viscous spreading \citep{concha-ramirez2019} are considered simultaneously. The truncation of the disks is calculated with expressions obtained from the similarity solutions of \citet{lynden-bell1974}.

In \citet{concha-ramirez2019} we investigated the effect of viscous growth and dynamical truncations on protoplanetary disks inside stars clusters. Both mechanisms were implemented with parametrized models. We showed that considering only viscous evolution and dynamical encounters yields circumstellar disks too large to be comparable to the compact disks observed in star forming regions. We argued that other processes must be at play to produce the observed disk size, mass, and accretion rate distributions. In the present work we expand our model by improving our description of the viscous growth of the disks, and by adding external photevaporation to the processes that can shrink them. In this way we look to better represent the environment of star forming regions. 

Instead of the parametrized description used in our previous work, we now modeled the circumstellar disks using the numerical code VADER \citep{krumholz2015}. This code solves the equations of angular momentum and mass transport in a thin, axisymmetric disk, providing us with a numerical tool to implement the viscous expansion that is less computationally expensive than SPH disks. While the dynamical truncations were still parametrized, the mass loss due to external photevaporation was calculated using the FRIED grid \citep{haworth2018}. This grid consists of pre-calculated mass loss rates for disks of different sizes and masses, immersed in several different external FUV fields. We used the Astrophysical Multipurpose Software Environment \citep[AMUSE\footnote{\url{http://amusecode.org}},][]{portegieszwart2019} framework to bring these codes together along with cluster dynamics and stellar evolution. All the code developed for the simulations, data analyses, and figures of this paper is available in a Github repository\footnote{ZENODO LINK}.


\section{Model}
\label{sec:model}

\subsection{Viscous growth of circumstellar disks}\label{viscous}
We implement the circumstellar disks with viscous evolution using the Viscous Accretion Disk Evolution Resource (VADER) by \citet{krumholz2015}. This is an open source viscous evolution code which uses finite-volume discretization to solve the equations of mass transport, angular momentum, and internal energy in a thin, axisymmetric disk. An AMUSE interface for VADER was developed in the context of this work and is available online\note{add footnote w/ link -- check if Martijn is ok with this!}.  

For the initial disk column density we use the standard disk profile introduced by \citet{lynden-bell1974}:

\begin{equation}\label{profile}
\Sigma(r, t=0) = \Sigma_0 \frac{r_c}{r} \exp\left(\frac{-r}{r_c}\right),
\end{equation}

\noindent
with

\begin{equation}
\Sigma_0 = \frac{m_d}{2 \pi {r_c}^2 \left(1 - \exp\left(-r_d/r_c\right)\right)},
\end{equation}

\noindent
where $r_c$ is the characteristic radius of the disk, $r_d$ and $m_d$ are the initial radius and mass of the disk, respectively, and $\Sigma_0$ is a normalization constant. Considering $r_c \approx r_d$ \citep{anderson2013}, the density profile of the disks takes the form:

\begin{equation}
\Sigma(r, t=0) \approx \frac{m_d}{2 \pi r_d \left(1 - \exp^{-1}\right)} \frac{\exp(-r/r_d)}{r}.
\end{equation}

This expression allows the disk to expand freely at the outer boundary while keeping the condition of zero torque at the inner boundary $r_i$ ($\Sigma(r_i)$ = 0).

To establish the radius of the disks we set the column density outside $r_d$ to a negligible value $\Sigma_{\mathrm{edge}} = \SI{e-12}{\gram\per\square\cm}$. 

The temperature profile of the disks is given by

\begin{equation}
T(r) = T_m \left(\frac{r}{\SI{1}{au}}\right)^{-1/2},
\end{equation}

\noindent
where $T_m$ is the midplane temperature at $\SI{1}{au}$. Based on \citet{anderson2013} we adopt $T_m = \SI{300}{\K}$.

Each disk is composed of a grid of 500 logarithmically spaced cells, in a range between $0.05$ and $\SI{5000}{au}$, with a Keplerian rotation profile. The fact that the size of the grid is much bigger than the disk sizes (which are intially around $\SI{100}{au}$, see section \ref{initdisks}) allows for the disks to expand freely without reaching the border of the grid. The mass flow through the outer boundary of the grid is set to zero in order to maintain the density $\Sigma_{\mathrm{edge}}$ needed to define the disk radius. The mass flow through the inner boundary of the grid is considered as accreted mass and added to the mass of the host star.

\subsection{Dynamical truncations}\label{truncations}
Between encounters, circumstellar disks evolve according to the numerical solutions of the transport equations used by VADER. A close encounter with another disk induces a discontinuity in this evolution. To modify the disks we calculate parametrized truncation radii. For two stars of the same mass, \citet{rosotti2014} approximated the truncation radius to a third of the encounter distance. To this we add the mass dependence of \citet{bhandare2016}, to have an expression for an encounter between stars of different masses. The truncation radius then takes the form:

\begin{equation}\label{truncationradius}
{r^\prime} = \frac{r_{enc}}{3}\left(\frac{M}{m}\right)^{0.5},
\end{equation}

\noindent 
where $M$ and $m$ are the masses of the encountering stars.

To implement truncations we first calculate the corresponding truncation radius of the encounter, according to equation \ref{truncationradius}. To define $r^\prime$ as the new disk radius we change the column density of all the disk cells outside $r^\prime$ to the lower value $\Sigma_{\mathrm{edge}}$, as described in section \ref{truncations}.

%Close dynamical encounters not only affect the radius of the affected disks, but also their mass. We followed \citet{portegieszwart2016} to define the amount of mass loss from each disk as:

%\begin{equation}
%dm = m_{d}\frac{r^{1/2} - R^{1/2}}{r^{1/2}}
%\end{equation}

%\noindent
%where $r$ and $R$ are the radii of the disks involved. We calculated the amount of mass that each disk accretes by:

%\begin{equation}\label{accmass}
%dm_{acc} = f \frac{m}{M + m} dm
%\end{equation}

%\noindent
%where $f \leq 1$ is a mass transfer efficiency factor. For our calculations we used $f = 1$. Both equations of mass accretion were applied symmetrically to each of the disks involved in the close encounter.

\subsection{External photoevaporation}\label{photoevaporation}
The amount of mass lost from disks as a result of external photoevaporation depends on the FUV luminosity of the bright stars in the cluster. This luminosity, together with the distance from each of the bright stars to the disks, is used to obtain the amount of radiation received by each disk. Below we explain what we consider to be bright stars and how we calculate the mass loss ratio.


\subsubsection{FUV luminosities}\label{FUVluminosities}
External photoevaporation of circumstellar disks in proximity to OB-type stars is dominated by FUV radiation \citep{guarcello2016,storzer1999,gorti2016a}. We follow \citet{adams2004} in defining the FUV band ranging from 6 eV up to 13.6 eV, or approximately from 912 Å to 2070 Å. Given the presence of spectral lines in this band, we use synthetic stellar spectra rather than relying on black body approximations. The spectral library used is UVBLUE \citep{rodriguez-merino2005}, for its high coverage of parameter space and high resolution, spanning the needed wavelength ranges. To make computations less expensive we use versions of the spectra with a resolving power five times lower than the full ones, which was sufficient for the calculations. 

The UVBLUE spectral library spans a three dimensional parameter space of stellar temperature, metallicity, and surface gravity. We used solar metallicity for all our stars. For the temperature and surface gravity, we chose the spectra that most closely resembled the zero age main sequence (ZAMS) values of each star, according to the parametrized stellar evolution code SeBa \citep{portegieszwart1996, toonen2012}. \note{should I add a more detailed explanation of this process?}

Using the ZAMS spectra, we calculated the flux and corresponding luminosity using the radius of the star. Given that we knew the masses of the stars, we built a relationship between stellar mass and FUV luminosity. This relation takes the shape of a segmented power law, as can be seen in Figure X. A similar fit was obtained by \citet{parravano2003}.
%Ask for more details of this process?

% FUV luminosity vs mass figure!

% Add table?

The mass range of our fit is $0.12 - 100 M_{\odot}$, however, we were only interested in the range $1.9 - 100 M_{\odot}$. As is further explained in section \ref{grid}, we only considered stars with masses higher that $1.9 M_{\odot}$ to be emitting FUV radiation, and $100 M_{\odot}$ was the upper limit for our stellar mass distribution. These bright stars were also subject to stellar evolution, which was implemented with SeBa through the AMUSE interface. The FUV luminosity of each bright star is calculated in every time step, after going through stellar evolution. The small, disked stars were not subject to stellar evolution.

\subsubsection{Mass loss ratio due to external photoevaporation}\label{grid}
To calculate the ratio of mass loss due to external photoevaporation on each disk we used the FRIED (Far-ultraviolet Radiation Induced Evaporation of Discs) grid developed by \citet{haworth2018}. The FRIED grid is an open access grid of calculations of externally evaporating circumstellar disks. It spans a six dimension parameter space consisting of disk sizes ($1 - 400 au$), disk masses ($~10^{-4} - 10^{2} M_{Jup}$), UV fields ($10 - 10^{4} G_0$), stellar masses ($0.05 - 1.9 M_{\odot}$) and disk outer surface densities. The seemingly three dimensional subspace of disk mass, edge density, and disk radius is in fact two dimensional, as any combination of disk radius and disk mass has only one edge density associated with it. Because of this, we only took into account a four dimensional grid of radiation field strength, host star mass, disk mass and disk radius.

Following the stellar mass ranges of the FRIED grid, we separated the stars in our clusters into two subgroups: \textit{bright stars} and \textit{small stars}. Bright stars are all stars with initial masses higher than $1.9 M_{\odot}$, while small stars have masses lower or equal than $1.9 M_{\odot}$. Only the small stars will have disks around them. The bright stars will be considered as only generating FUV radiation and affecting the disks around the smaller stars. In this way we made sure that we stayed within the stellar mass limits of the grid. Low mass stars ($\lesssim 1 M_\odot$) have a negligible UV flux \citep{adams2006}, so this approximation holds well for our purposes. Calculation of the FUV radiation emitted by the bright stars is further explained in section \ref{FUVluminosities}. In terms of gravitational evolution, both star groups were indistinguishable and they evolved together normally in the gravity code. These star subgroups were considered only for the calculation of FUV radiation and mass loss.

The FRIED grid allows to take a circumstellar disk with a specific mass, size, and density, around a star with a certain mass, and from the FUV radiation that it receives, obtain the photoevaporation mass loss. However, the parameters of our disks not always exactly matched the ones in the grid. We performed an interpolation over the grid to calculate the mass losses of the disks in our simulations. Because of computational constraints, we performed the interpolations on a subspace of the grid, such that it contains at least one data point above and below the phase space point of the disk in each dimension. In \note{Appendix X} we show that such a subgrid is enough to perform the interpolation.

\subsubsection{Disk truncation due to photoevaporation}\label{omnomnom}
Once the mass lost due to photoevaporation was calculated for every disk, the disks had to be truncated at a point that coincided with the amount of mass lost in the process. For this, we move outside-in through the disk and remove mass from each cell, by turning its column density to our limit value $\Sigma_{\mathrm{edge}}$. We stopped at the point where the mass removed from the disk is equal to the calculated mass loss due to photoevaporation.

\subsubsection{Summary of the photoevaporation truncation process}
In summary, in every simulation time step, the following procedures were carried out to simulate disk truncation due to photoevaporation:
\note{should I write the following in pseudocode format to make it clearer?}
\begin{enumerate}
\item For each bright star we calculated its FUV luminosity $L_{FUV}$ as described in section \ref{FUVluminosities}.
\item For each small star we calculated the distance to each bright star, $d$.
\item Using $L_{FUV}$ and $d$ we calculated the amount of FUV radiation, $\ell_{FUV}$, received by the small star. \note{check for proper symbols}
\item Using $\ell_{FUV}$ and the stellar mass, disk mass, and disk size of the small star, we built a subgrid of the FRIED grid and interpolated on it to calculate $\dot{M}_FUV$, the photoevaporation mass loss.
\item We subtracted that amount of mass from the disk mass.
\end{enumerate}


\subsection{Evolution of the disks}
For each simulation we did the following:

-Start a Plummer sphere with N number of stars and virial size N

-Select all the stars under 1.9 MSun (because of the FRIED grid parameter space). To all these stars we gave circumstellar disks.

-For all the stars > 1.9 MSun (a.k.a. massive stars) we implemented stellar evolution.

-We used the UVBLUE spectral library to calculate the FUV luminosity of each of these massive stars.

-On each timestep we calculate the positions of the stars, and the radiation that each small stars gets from all the massive stars, in the distance

-This gives us the total photoevaporation mass loss rate for each small star

-We update the size of the circumstellar disk according to the viscous growth and to the photoevaporation mass loss rate


We consider the disks to be completely evaporated when they have lost 99\% of their initial mass, based on \citet{anderson2013}.

\section{Results}
\label{sec:results}

\subsection{Initial conditions}
\subsubsection{Cluster}
We run X simulations with N stars each located in Plummer (King?) spheres of Z size. We run M simulations lasting for 10 million years each. 

We then plot the cumulative distributions of the protoplanetary disks. To calculate the size of the protoplanetary disks we take the mass inside a characteristic radius. We set this radius to 95\% to make it similar to observations (check reference on my previous paper).

\subsubsection{Disks}\label{initdisks}
The initial sizes of our circumstellar disks were given by:

\begin{equation}
r_d(t=0) = R'\left(\frac{M_*}{M_\odot}\right)^{0.5}
\end{equation}

\noindent
where $R'$ is a constant. The youngest circumstellar disks observed to date have sizes that range from $\sim\SI{30}{au}$ \citep[e.g.][]{lee2018} to $\sim120-\SI{180}{au}$ \citep[e.g.][]{murillo2013,vanthoff2018}. Based on this we chose the value $R' = \SI{100}{au}$, which for our mass range $0.05-1.9 M_\odot$ for stars with disks yields initial sizes between $\sim\SI{22}{au}$ and $\sim\SI{137}{au}$.

\subsection{Photoevaporation vs dynamical truncations}

\subsection{Disk lifetimes}

\subsection{Comparison with observations}
\note{or should this go in the discussion?}

\section{Discussion}
\label{sec:discussion}

\subsection{Caveats}
Our model encompassed a simplified implementation of the processes inside young star forming regions. Several mechanisms and circumstances were not considered for our calculations. Regarding photoevaporation, our model ignores the different regimes in which this can occur. External photoevaporation can ensue in super and subcritical regimes. These division is established in terms of the critical radius $r_g$ at which the sound speed of the gas equals the escape speed from the disk \citep{adams2004}:

\begin{equation}
r_g = \frac{GM_*\left\langle\mu\right\rangle}{kT} \approx \SI{100}{au} \left(\frac{T}{\SI{1000}{K}}\right)^{-1}\left(\frac{M_*}{\SI{1}{M_\odot}}\right).
\end{equation}

The photoevaporation flows are different for disks with radius $r_{disk} < r_g$ (subcritical) and $r_{disk} > r_g$ (supercritical). \citet{adams2004} showed that the mass outflow rates are different for each of these cases, and that many disks fall in the subcritical regime. They also showed that even for disks where photoevaporation is suppressed because they have reached the subcritical size, significant mass loss still takes place as long as $r_{disk}/r_g \gtrsim 0.1-0.2$. This difference in photoevaporation regimes was not considered in our implementation, where all mass is equally removed independent of the radius of the disk where is it located. Also, our implementation only considered photoevaporation mass loss as working from the outside in, and did not include mass that can be removed from the surface of the disk at smaller radii. 

Furthermore, our implementation completely ignored the photoevaporation caused by the radiation of the disk's host star. \note{elaborate} \citep{fatuzzo2008}

We also did not differentiate between FUV and EUV radiation. This approximation is valid since it is unlikely that EUV radiation drives the mass loss of disks \citep{owen2012}. FUV photons have a much larger penetrating depth than EUV, and the former generate an ionization front several radii away from the disk surface which absorbs the incident EUV flux \citep{adams2004}. 

-We ignore dust growth and ignore different components for dust and gas disks 

-We ignore binaries (cite cool paper on circumbinary stuff)

-We ignore gas in the star forming regions

Stellar evolution was used only as a way to obtain the FUV luminosity of the massive stars at each moment in the evolution of the clusters. Outcomes of stellar evolution, such as stellar winds and supernovae explosions, were not considered in the simulations, but they could have a significant effect on the development of circumstellar disks inside star clusters \citep{close2017,portegieszwart2018,pelupessy2012}.

Disk winds could also be an important cause of mass loss on the disks. While some hints of disk winds have been observed, mass loss rates related to this process are still unconstrained due to instrumental restrictions \citep{pascucci2018}. It is expected that new generations of instruments such as ngVLA \note{citation} will help determine how critical these winds can be to disk evolution.

\subsection{Comparison with observations?}

\section{Summary}
\label{sec:summary}


% Don't change these lines
\bsp	% typesetting comment
\bibliographystyle{mnras}
\bibliography{references}
\label{lastpage}
\end{document}

% End of mnras_template.tex
