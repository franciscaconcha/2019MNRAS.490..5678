% mnras_template.tex 
%
% LaTeX template for creating an MNRAS paper
%
% v3.0 released 14 May 2015
% (version numbers match those of mnras.cls)
%
% Copyright (C) Royal Astronomical Society 2015
% Authors:
% Keith T. Smith (Royal Astronomical Society)

% Change log
%
% v3.0 May 2015
%    Renamed to match the new package name
%    Version number matches mnras.cls
%    A few minor tweaks to wording
% v1.0 September 2013
%    Beta testing only - never publicly released
%    First version: a simple (ish) template for creating an MNRAS paper

%%%%%%%%%%%%%%%%%%%%%%%%%%%%%%%%%%%%%%%%%%%%%%%%%%
% Basic setup. Most papers should leave these options alone.
\documentclass[fleqn,usenatbib]{mnras}

% MNRAS is set in Times font. If you don't have this installed (most LaTeX
% installations will be fine) or prefer the old Computer Modern fonts, comment
% out the following line
\usepackage{newtxtext,newtxmath}
% Depending on your LaTeX fonts installation, you might get better results with one of these:
%\usepackage{mathptmx}
%\usepackage{txfonts}

% Use vector fonts, so it zooms properly in on-screen viewing software
% Don't change these lines unless you know what you are doing
\usepackage[T1]{fontenc}
\usepackage{ae,aecompl}
\usepackage[utf8]{inputenc}


%%%%% AUTHORS - PLACE YOUR OWN PACKAGES HERE %%%%%

% Only include extra packages if you really need them. Common packages are:
\usepackage{graphicx}	% Including figure files
\usepackage{amsmath}	% Advanced maths commands
\usepackage{amssymb}	% Extra maths symbols
\usepackage{multirow}
\usepackage{txfonts}
\usepackage[utf8]{inputenc}
\usepackage{float}
\usepackage{subcaption}
\usepackage[T1]{fontenc}
\usepackage{microtype}

\usepackage{subfiles} % to split the paper into different files

\usepackage[decimalsymbol=., expproduct=times, separate-uncertainty=true, multi-part-units=single]{siunitx} % SI units 

%%%%%%%%%%%%%%%%%%%%%%%%%%%%%%%%%%%%%%%%%%%%%%%%%%

%%%%% AUTHORS - PLACE YOUR OWN COMMANDS HERE %%%%%

% Please keep new commands to a minimum, and use \newcommand not \def to avoid
% overwriting existing commands. Example:
%\newcommand{\pcm}{\,cm$^{-2}$}	% per cm-squared

%%%%%%%%%%%%%%%%%%%%%%%%%%%%%%%%%%%%%%%%%%%%%%%%%%

%%%%%%%%%%%%%%%%%%% TITLE PAGE %%%%%%%%%%%%%%%%%%%

% Title of the paper, and the short title which is used in the headers.
% Keep the title short and informative.
\title[External photoevaporation of disks in young star clusters]{External photoevaporation shapes the distributions of circumstellar disks in young star clusters}

% The list of authors, and the short list which is used in the headers.
% If you need two or more lines of authors, add an extra line using \newauthor
\author[Concha-Ramírez et al.]{
Francisca Concha-Ramírez\thanks{E-mail: fconcha@strw.leidenuniv.nl},
Martijn Wilhelm,
Simon Portegies Zwart
\\
% List of institutions
% List of institutions
%$^{1}$
Leiden Observatory, Leiden University, PO Box 9513, 2300 RA Leiden, The Netherlands\\
}

% These dates will be filled out by the publisher
\date{Accepted XXX. Received YYY; in original form ZZZ}

% Enter the current year, for the copyright statements etc.
\pubyear{2018}

% Don't change these lines
%\hypersetup{draft}
\begin{document}
\label{firstpage}
\pagerange{\pageref{firstpage}--\pageref{lastpage}}
\maketitle

% Abstract of the paper
\begin{abstract}
There are different processes which may affect the evolution of circumstellar disks inside young star clusters and star forming regions. 
TESTING GIT STUFF!!!
\end{abstract}

% Select between one and six entries from the list of approved keywords.
% Don't make up new ones.
\begin{keywords}
key 1 key 2
\end{keywords}

\section{Introduction}
\label{sec:introduction}

Circumstellar disks are formed shortly after star formation. During their early years, they are immersed in the environment of star formation; that is, surrounded by left over gas and prone to dynamical interactions with other, nearby stars. 
The environment in which circumstellar disks are born can have an important impact on their evolution. Some of the mechanisms affecting their development correspond to their intrinsic viscous growth, as well as external agents such as truncation by dynamical encounters, external photevaporation, face-on accretion, and ram pressure stripping. 

It is interesting to look into the evolution of circumstellar disks inside star cluster. Our very own Sun was born within a star cluster, so understanding how the cluster environment affects the evolution of the disks can help us understand the origins of our own Solar System. Previous work has been done to study the effects of viscous growth (me, Rossoti2014) and dynamical truncations on circumstellar disks inside star clusters (spz, pfalzner). This has been done both semi-analytically and numerically. 

Work has also been done in studying both the internal and external photoevaporation of circumstellar disks. Inside star clusters, it is likely that the disks will be affected by radiation from bright stars. Work by ... and ... shows that disks are especially sensitive to FUV radiation from OB stars.

This problem is very expensive to run numerically...

In this work we present a semi analytical modelling of protoplanetary disks subject to viscous growth, dynamical truncations, and external photevaporation. We expand on the work by ME2018, where viscous growth and truncations were modelled semi analytically. We use the developments of X and Y to parametrize external photevaporation as a mass loss rate on the disk.

Our previous work showed that considering only viscous growth and dynamical truncations yields disks that are too large to be compared with current observations of protoplanetary disks. Dynamical truncations by themselves are not important enough to account for the sizes of the observed disks. There must be some other mechanisms that generates the compact disks we observe in star clusters and young star forming regions.


\section{Model}
\label{sec:model}

The viscous growth and the dynamical truncations were modeled in the same way as in FCR2018. In sections 1 and 2 we present a summary of the implementations. In section 3 we present the new implementations for photevaporation developed for the present work.

\subsection{Viscous growth of disks}

We base our description of the disks on the similarity solutions of Lynden-Bell and Pringle 1974, and the implementation of Hartmann1998. We characterize our disks by three values: their characteristic radius, their mass, and their accretion rate onto their host star.

We use values of gamma=1 and alpha=1E-4 as argued in FCR2018. This gives us realistic viscous timescales for the disks.

In this occasion we consider the characteristic radius as the radius inside which 95\% of the disk's mass resides. This, to make a closer comparison with observations (cite same reference than in my previous paper).

\subsection{Dynamical truncations}
Dynamical truncations are the same as in the previous paper. Defined by SPZ. We also use the mass from X.

\subsection{Photoevaporation}
External photoevaporation of protoplanetary disks is dominated by FUV radiation. We follow \citet{adams_photoevaporation_2004} in defining the FUV range as being between 


We modeled only the external photoevaporation of the disk. We consider internal photoevaporation to be negligible for our analyses.

We model external photoevaporation as a mass loss that affects the stellar disks. This mass loss was calculated using the FRIED grid. The FRIED grid is a grid which calculates FUV photoevaporative mass losses for a big parameter space of disks. 

\subsubsection{Getting the FUV luminosity of the stars}
We need to calculate the FUV luminosity of each of the massive stars. For this, we use the UVBLUE spectral library. We choose all our stars to have solar metallicity. We take the ZAMS temperature of each massive star.

We calculate the ZAMS luminosity of the stars. From this we choose a spectrum, and we fit it. This allows us to get a stellar mass vs FUV luminosity relationship, which is what we use to calculate the FUV luminosity of each star all through our simulation. The fit can be seen in Figure X. The values obtained for the fit can be seen in Table Y.

External photoevaporation of the disks is a function not only of the disk properties, but also of the radiation field in which the disk is immersed. tion not only of the disk properties, but also of the radiation field in which the disk is immersed. 

\subsection{Evolution of the disks}
For each simulation we did the following:

-Start a Plummer sphere with N number of stars and virial size N

-Select all the stars under 1.9 MSun (because of the FRIED grid parameter space). To all these stars we gave circumstellar disks.

-For all the stars > 1.9 MSun (a.k.a. massive stars) we implemented stellar evolution.

-We used the UVBLUE spectral library to calculate the FUV luminosity of each of these massive stars.

-On each timestep we calculate the positions of the stars, and the radiation that each small stars gets from all the massive stars, in the distance

-This gives us the total photoevaporative mass loss rate for each small star

-We update the size of the circumstellar disk according to the viscous growth and to the photoevaporative mass loss rate

\section{Results}
\label{sec:results}

\subsection{Initial conditions}
We run X simulations with N stars each located in Plummer (King?) spheres of Z size. We run M simulations lasting for 10 million years each. 

We then plot the cumulative distributions of the protoplanetary disks. To calculate the size of the protoplanetary disks we take the mass inside a characteristic radius. We set this radius to 95\% to make it similar to observations (check reference on my previous paper).

\section{Discussion}
\label{sec:discussion}

\subsection{Caveats}
-We ignore the subsonic and supersonic photoevaporation regimes 

-We ignore dust growth and ignore different components for dust and gas disks 

-We ignore binaries (cite cool paper on circumbinary stuff)

-We ignore gas in the star forming regions

-We ignore actual stellar evolution!!!

-We could maybe actually evolve the disks instead of parametrizing them

\subsection{Comparison with observations}

\section{Summary}
\label{sec:summary}

We de best

% Don't change these lines
\bsp	% typesetting comment
\bibliographystyle{mnras}
\bibliography{references}
\label{lastpage}
\end{document}

% End of mnras_template.tex